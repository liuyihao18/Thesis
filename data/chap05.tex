% !TeX root = ../thuthesis-example.tex

\chapter{实验结果分析}

本章将对远距离多目标运动特征感知系统原型进行实验验证与性能分析。
实验主要包含三部分,分别为模块性能测试、室外远距离多目标感知实验以及室内多目标感知实验。
在每部分实验中,将介绍实验环境与设置,并对实验结果进行分析。

\section{模块性能测试}

该部分实验旨在测试系统各模块的性能表现,包括信号提取模块、偏移消除模块和目标感知模块等,以验证本文提出方法的有效性。
最后使用定向天线进行端到端实验,证明系统在实际应用中的可行性。

\subsection{改进汉明窗}

本小节仿真测试了改进汉明窗在缓解远近效应中的性能表现。
仿真程序使用MATLAB编写,模拟了极端场景中的远近效应对信号处理的影响。
在这个场景中,存在一个近距离标签和一个远距离标签。
近距离标签信号的振幅为1,频率为10Hz,而远距离标签信号的振幅为0.01,频率为14Hz。
也就是说,近距离标签信号的强度是远距离标签信号强度的10000倍(40dB)。
标签信号长度为1s,满足正交性约束$\left|f_2 - f_1\right|T \gg 1$,理论上可以区分出两个标签信号。
然而,由于两个标签信号强度差异过大,近距离标签信号的旁瓣掩盖了远距离标签信号的主瓣。
图~\ref{fig:window-result}~展示了对信号使用改进汉明窗后的频谱结果,此时可以明确地区分出两个信号。

\begin{figure}[htbp]
    \centering
    \includegraphics[width=0.66\linewidth]{chap05/window_result.pdf}
    \caption{不同加窗方法下的频谱结果比较}
    \label{fig:window-result}
\end{figure}

\subsection{收发端相位校正}

本小节
%\footnote{收发端相位校正和标签频率漂移消除方法二者缺一不可,因此在测试收发端相位校正方法的时候,使用了标签频率漂移消除方法。}
通过实验测试了收发端相位校正方法在消除载波频率偏移(CFO)、采样时间偏移(STO)和采样频率偏移(SFO)中的性能表现。
本文提出通过激励信号与反向散射信号之间的内在联系来消除这些偏移。
实验使用软件定义无线电设备HackRF One作为发送端和接收端,实验环境为室内实验环境,发送端与接收端之间的距离约1米,频移标签位于发送端与接收端之间,频移标签保持静止状态。
在该实验里,收发端采样率为1MHz,中心频率为915MHz,基带啁啾信号带宽为100KHz,长度为20ms,频移标签频率约253KHz。

\begin{figure}[htbp]
    \centering
    \includegraphics[width=0.66\linewidth]{chap05/cfo_compensation.pdf}
    \caption{收发端相位校正实验结果}
    \label{fig:cfo-compensation}
\end{figure}

为了更好地展示收发端相位校正方法的效果,本文将其与传统的基于频域峰值检测的CFO估计补偿方法~\cite{jiang2021long}进行了对比。
实验结果如图~\ref{fig:cfo-compensation}~所示。
图中橙色曲线是传统方法的结果,而蓝色曲线是本文提出方法的结果。
可以看到,传统方法由于CFO的估计精度有限且没有考虑STO和SFO的影响,导致频率估计结果存在较大偏差。
而本文提出的方法通过信号相除有效地消除了CFO、STO和SFO的影响,残余相位仅剩0.013rad,几乎不会对后续的目标感知产生影响。

\subsection{标签频率漂移消除}

本小节
%\footnote{收发端相位校正和标签频率漂移消除方法二者缺一不可,因此在测试标签频率漂移消除方法的时候,使用了收发端相位校正方法。}
通过实验测试了标签频率漂移消除方法在修正由标签频率漂移引起的相位误差中的性能表现。
本文提出利用反向散射信号双边带的信息,结合标签调制的基本特性来修正频率漂移误差。
实验设置与上一小节一致,发送端和接收端使用HackRF One软件定义无线电设备,频移标签保持静止状态。

\begin{figure}[htbp]
    \centering
    \includegraphics[width=0.66\linewidth]{chap05/tag_freq.pdf}
    \caption{标签双边带信号频率测量结果}
    \label{fig:tag-freq}
\end{figure}

首先,本小节验证了标签调制的基本特性——双边带信号频率(相位)变化方向相反。
图~\ref{fig:tag-freq}~展示了标签双边带信号的频率测量结果,可以看到,双边带信号的频率变化方向确实相反。
这证明了标签频率漂移消除方法的基本假设是成立的,从而本文提出的利用标签信号双边带的信息来修正相位偏移的方法是可行的。

\begin{figure}[htbp]
    \centering
    \includegraphics[width=0.66\linewidth]{chap05/tag_drift.pdf}
    \caption{标签频率漂移消除实验结果}
    \label{fig:tag-drift}
\end{figure}

图~\ref{fig:tag-drift}~展示了标签频率漂移消除实验的结果。
为了让读者有更直观的理解,本文将其与朴素的基于标签频率估计的相位补偿方法进行了对比。
图中橙色曲线是朴素方法得到的结果,而蓝色曲线是本文提出方法得到的结果。
可以看到,利用朴素方法补偿的信号相位很快就产生了较大偏移,使得感知完全无法进行。
而本文提出的方法有效地解决了标签频率漂移问题,补偿后信号相位几乎保持不变。
值得注意的是,由于标签成本低,标签频率漂移对相位的影响远超上一小节的收发端频率偏移对相位的影响,因此标签频率漂移消除方法在实际应用中尤为重要。
