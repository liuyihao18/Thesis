% !TeX root = ../thuthesis-example.tex

\chapter{正交感知模型}

\section{基础知识介绍}

在本节中,将首先介绍无线信号的基础知识,然后介绍与本文相关的啁啾信号和反向散射信号的原理,接着介绍正交性的概念,最后介绍一种利用相量理解信号的方式。

\subsection{无线信号原理}

无线信号是通过电磁波在空间中传播的信息载体,其传输过程包括信号的生成、调制、传播、接收和解调等环节。
无线信号通常使用IQ调制,将原始基带信号分解为IQ两路分量分别调制在相位相差90$^\circ$的载波上进行传输。
其中,I分量表示同相分量(In-Phase),Q分量表示正交分量(Quadrature)。
虽然在现实世界中不会出现“虚数信号”,但是在信号处理领域中,仍然可以使用复数工具来简化计算过程。
本文不具体展开IQ调制的原理,但在下面的章节中会使用复数工具来描述信号。
对于最简单的单频信号,可以表示为:
\begin{equation}
    s(t) = A e^{j(2\pi f t + \theta)}
    \label{eq:single-tone}
\end{equation}

\subsection{啁啾信号原理} \label{sec:chirp}

LoRa广域网技术中使用啁啾信号作为基带信号来提高信号的抗干扰能力与传播距离。
啁啾信号采用线性调频(Chirp Spreading Spectrum, CSS)的方式来编码信息,每一个符号的频率随时间线性变化。

\begin{figure}
    \centering
    \includegraphics[width=0.8\linewidth]{chap02/chirp.pdf}
    \caption{啁啾信号}
    \label{fig:chirp}
\end{figure}

如图~\ref{fig:chirp}~所示,最基本的啁啾信号形式为上啁啾符号,其频率从$-\frac{BW}{2}$线性增长到$\frac{BW}{2}$,其中$BW$是啁啾信号的带宽。
下啁啾符号是上啁啾符号的共轭形式(或者称为时间反转形式),其频率从$\frac{BW}{2}$线性下降到$-\frac{BW}{2}$。
而一般的数据啁啾符号以初始频率$f_0$编码数据,其频率先从$f_0$线性增长到$\frac{BW}{2}$,然后卷绕回$-\frac{BW}{2}$,再从$-\frac{BW}{2}$线性增长到$f_0$。
写成公式是:
\begin{equation}
    s_{\text{up}}(t) = e^{j2\pi\left(-\frac{BW}{2} t + \frac{1}{2}kt^2\right)},\ t \in [0, T)
\end{equation}
\begin{equation}
    s_{\text{down}}(t) = e^{-j2\pi\left(-\frac{BW}{2} t + \frac{1}{2}kt^2\right)},\ t \in [0, T)
\end{equation}
\begin{equation}
    s(t) =
    \begin{cases}
        e^{j2\pi\left(f_0 t + \frac{1}{2}kt^2\right)},\ t \in [0, t_\text{wrap}) \\
        e^{j2\pi\left(-\frac{BW}{2} (t - t_\text{wrap}) + \frac{1}{2}k(t-t_\text{wrap})^2\right)},\ t \in [t_\text{wrap}, T)
    \end{cases}
\end{equation}

其中,$k=\frac{BW}{T}$是频率变化的速率,$T$是啁啾符号的持续时间。
LoRa使用一个参数来控制啁啾符号的形状和初始频率,称为扩频因子(Spreading Factor, SF),其定义为:
\begin{equation}
    SF = \log_2(BW \cdot T)
\end{equation}

扩频因子决定了$f_0$的可能取值$f_n$,也决定了卷绕发生时间$t_\text{wrap}$的可能取值$t_n$,即:
\begin{equation}
    f_n = -\frac{BW}{2} + n \cdot \frac{BW}{2^{SF}},\ n = 0, 1, ..., 2^{SF}-1
\end{equation}
\begin{equation}
    t_n = T - n \cdot \frac{T}{2^{SF}},\ n = 0, 1, ..., 2^{SF}-1
\end{equation}

通常情况下,扩频因子越大,啁啾符号的持续时间越长,信号的抗干扰能力越强,但传输速率越低。
啁啾信号的解调过程是把接收到的啁啾信号与下啁啾信号相乘,将其转化为单频信号,然后通过傅里叶变换提取出该单频信号的初始频率$f_0$:
\begin{equation}
    \begin{aligned}
        r(t) & = s(t) \cdot s_{\text{down}}(t)                               \\
             & = \begin{cases}
                     e^{j2\pi (f_0 + \frac{BW}{2}) t},\ t \in [0, t_\text{wrap}) \\
                     e^{j2\pi (f_0 - \frac{BW}{2}) t},\ t \in [t_\text{wrap}, T)
                 \end{cases}
    \end{aligned}
\end{equation}

得益于啁啾信号的抗干扰特性,使用啁啾信号进行感知的系统在低信噪比环境下仍然能够有效地提取目标信息。
本文也采用啁啾信号作为无线感知系统的基带信号,利用其优良的特性来增加无线感知的范围并提高无线感知的精度。

\subsection{反向散射信号原理} \label{sec:backscatter}

\begin{figure}
    \centering
    \includegraphics[width=0.85\linewidth]{chap02/absorb.pdf}
    \caption{处于吸收状态的标签}
    \label{fig:absorb}
\end{figure}

\begin{figure}
    \centering
    \includegraphics[width=0.85\linewidth]{chap02/reflect.pdf}
    \caption{处于反射状态的标签}
    \label{fig:reflect}
\end{figure}

反向散射信号的基本原理是标签通过射频开关改变电路的接入阻抗,从而改变天线的反射系数来实现对信号的调制。
如图~\ref{fig:absorb}~所示,当射频开关处于OFF状态时,接入阻抗为电路阻抗,标签处于吸收状态;如图~\ref{fig:reflect}~所示,当射频开关处于ON状态时,接入阻抗为0,标签处于反射状态。
用数学公式表示的话,标签反射信号与入射信号间的关系为:
\begin{equation}
    S_{\text{out}} = \frac{Z_a-Z_c}{Z_a+Z_c} S_{\text{in}} = \sigma S_{\text{in}}
\end{equation}

其中,$S_{\text{in}}$和$S_{\text{out}}$分别是入射信号和反射信号,$Z_a$是天线阻抗,$Z_c$是接入阻抗,$\sigma$是天线的反射系数。

如果射频开关切换频率比较低,那么标签将表现为以开关键控方式调制信号。
如果射频开关切换频率比较高,那么标签将表现为以频移键控方式控调制信号。
此时,相当于将开关频率$f_{\text{tag}}$的方波信号与入射信号相乘:
\begin{equation}
    S_{\text{out}}(t) = Square\left(A_{\text{tag}}, f_{\text{tag}}, t\right) \cdot S_{\text{in}}(t)
\end{equation}

其中,常常用方波信号的一次谐波来近似表示该方波信号:
\begin{equation}
    Square(A_{\text{tag}}, f_{\text{tag}}, t) \approx A_{\text{tag}}\cos\left(2\pi f_{\text{tag}} t + \psi_{\text{tag}}\right)
\end{equation}

根据欧拉公式,该信号可以进一步分解为两个频率分别为$f_{\text{tag}}$和$-f_{\text{tag}}$的正弦波:
\begin{equation}
    A_{\text{tag}} \cos(2\pi f_{\text{tag}} t + \psi_{\text{tag}}) = \frac{A_{\text{tag}}}{2}\left(e^{j\left(2\pi f_{\text{tag}} t + \psi_{\text{tag}}\right)} + e^{-j\left(2\pi f_{\text{tag}} t + \psi_{\text{tag}}\right)}\right)
\end{equation}

于是,反射信号可以表示为:
\begin{equation}
    \begin{aligned}
        S_{\text{out}}(t)
         & \approx \frac{A_{\text{tag}}}{2}\left(e^{j\left(2\pi f_{\text{tag}} t + \psi_{\text{tag}}\right)} + e^{-j\left(2\pi f_{\text{tag}} t + \psi_{\text{tag}}\right)}\right) S_{\text{in}}(t)                         \\
         & = \frac{A_{\text{tag}}}{2} e^{j\left(2\pi f_{\text{tag}} t + \psi_{\text{tag}}\right)} S_{\text{in}}(t) + \frac{A_{\text{tag}}}{2} e^{-j\left(2\pi f_{\text{tag}} t + \psi_{\text{tag}}\right)} S_{\text{in}}(t) \\
         & = S_{\text{out}}^{\text{upper}}(t) + S_{\text{out}}^{\text{lower}}(t)
    \end{aligned}
\end{equation}

这就是频移反向散射标签双边带信号的原理,注意到反射信号的双边带源自同一个方波调制信号,因此其相位特征呈现出对称性,在之后的章节中,将利用该性质解决标签频率漂移的问题。

\subsection{正交性原理}

在信号处理领域中,两个信号如果满足其内积为零,那么这两个信号被称为正交信号。
定义两个信号的内积运算为共轭相乘的归一化积分:
\begin{equation}
    \langle f(t), g(t) \rangle = \frac{1}{T}\int_{0}^{T}f(t)\overline{g(t)}\ \mathrm{d}t
\end{equation}

对于频率不同的两个单频信号,它们的积分可以计算为:
\begin{equation}
    \begin{aligned}
        \langle e^{j2\pi f_1 t}, e^{j2\pi f_2 t} \rangle
         & = \frac{1}{T}\int_{0}^{T} e^{j2\pi f_1 t} e^{-j2\pi f_2 t} \mathrm{d}t \\
         & = \frac{1}{T}\int_{0}^{T} e^{j2\pi (f_1 - f_2) t} \mathrm{d}t          \\
         & = \frac{1}{T}\frac{e^{j2\pi (f_1 - f_2) T} - 1}{j2\pi (f_1 - f_2)}     \\
    \end{aligned}
\end{equation}

考虑该积分的模长:
\begin{equation}
    \left|\langle e^{j2\pi f_1 t}, e^{j2\pi f_2 t} \rangle\right| = \left|\frac{\sin(\pi (f_1 - f_2) T)}{\pi (f_1 - f_2)T}\right|
\end{equation}

当$f_2 - f_1 = \frac{n}{T}, n \in \mathbb{Z} \setminus \{0\}$时,该积分的模长严格为0;当$|f_2 - f_1| T \gg 1$时,该积分的模长近似为0。
由此可以得到,\textbf{持续时间为$T$、频率差距远大于$\frac{1}{T}$的两个单频信号彼此正交}。

考虑接收端收到两个不同频率单频信号的情况,设两个单频信号分别为:
\begin{equation}
    s_1(t) = A_1 e^{j(2\pi f_1 t + \theta_1)}
\end{equation}
\begin{equation}
    s_2(t) = A_2 e^{j(2\pi f_2 t + \theta_2)}
\end{equation}

那么此时接收端收到的信号为两个单频信号的叠加:
\begin{equation}
    r(t) = s_1(t) + s_2(t) = A_1 e^{j(2\pi f_1 t + \theta_1)} + A_2 e^{j(2\pi f_2 t + \theta_2)}
\end{equation}

若接收端想要从接收到的信号中提取出某一个单频信号的幅度和相位信息,可以通过将其与该单频信号相同频率的信号的内积运算来实现:
\begin{equation}
    \begin{aligned}
        h_1 & = \langle r(t), e^{j2\pi f_1 t} \rangle
        = \frac{1}{T}\int_{0}^{T} r(t) e^{-j2\pi f_1 t} \mathrm{d}t                                                          \\
            & = \frac{1}{T}\int_{0}^{T} \left( A_1 e^{j\theta_1} + A_2 e^{j(2\pi(f_2 - f_1)t + \theta_2)}\right) \mathrm{d}t \\
            & = A_1 e^{j\theta_1} + \frac{1}{T}A_2 e^{j\theta_2} \int_{0}^{T} e^{j2\pi(f_2 - f_1)t} \mathrm{d}t              \\
            & \approx A_1 e^{j\theta_1}
    \end{aligned}
    \label{eq:orthogonal_extract}
\end{equation}

其中,最后一步近似成立的条件仍然是$|f_2 - f_1| \gg \frac{1}{T}$,也就是说,\textbf{持续时间为$T$、频率差距远大于$\frac{1}{T}$的两个单频信号可以在接收端被区分开来。}
同样地,接收端提取信号$s_2(t)$的幅度和相位信息可以表示为:
\begin{equation}
    \begin{aligned}
        h_2 & = \langle r(t), e^{j2\pi f_2 t} \rangle
        = A_2 e^{j\theta_2}
    \end{aligned}
\end{equation}

在实际的信号收发中,信号并非以连续函数的形式存在,而是以离散采样点的形式存在。
设采样频率为$fs$,则实际发送的信号可以表示为:
\begin{equation}
    x[n] = A e^{j2\pi \frac{k}{N}n}
\end{equation}

其中,$N$是采样点数,$k = \frac{f_1}{fs}N$是归一化的频率。
此时,内积运算定义为:
\begin{equation}
    \langle \vec{a}, \vec{b} \rangle = N\sum_{n=0}^{N-1}a_n\overline{b_n}
\end{equation}

类似地,可以得到两个不同频率单频信号的正交性结论,即当$|k_2 - k_1| \gg 1$时,两个单频信号彼此正交。

如果理解为希尔伯特空间,则此空间的基向量可以写作:
\begin{equation}
    \vec{e}_k=\left(\frac{1}{N}e^{j2\pi\frac{k}{N}\cdot 0}, \frac{1}{N}e^{j2\pi\frac{k}{N}\cdot 1}, ..., \frac{1}{N}e^{j2\pi\frac{k}{N}\cdot (N-1)}\right),\ k = 0, 1, ..., N-1
\end{equation}

任意信号可以表示为:
\begin{equation}
    x[n] = \frac{1}{N}\sum_{k=0}^{N-1}c_ke^{j2\pi \frac{k}{N}n},\ n = 0, 1, ..., N-1
\end{equation}

于是,信号$x[n]$在基向量$\vec{e}_k$上的投影系数$c_k$可以表示为:
\begin{equation}
    \begin{aligned}
        c_k & = \langle x[n], e^{j2\pi \frac{k}{N}n} \rangle
        = N\sum_{n=0}^{N-1}x[n]e^{-j2\pi \frac{k}{N}n}       \\
            & = \FFT[x[n]](k), \ k=0,1,...,N-1
    \end{aligned}
\end{equation}

这也是离散傅里叶变换分离正交信号的原理。

在本文提出的感知系统当中,利用了多种信号正交的性质,最终实现了远距离多目标运动特征感知系统。
具体来说,本文利用了以下四种正交性质:

(1)\textbf{不同参数的啁啾信号彼此正交:}在对啁啾信号解调时,不同参数的啁啾信号被转换成不同频率的单频信号,而这些单频信号彼此正交,可以被区分开来。这个性质使得一个接收端可以与多个发送端协作,而多个发送端之间互不干扰。

(2)\textbf{反向散射信号与激励信号彼此正交:}反向散射信号过频移调制后,其频率远离了激励信号的频率,从而反向散射信号与激励信号彼此正交,可以被区分开来。这个性质使得反向散射信号可以在强直射信号的干扰下被有效提取出来,也使得利用激励信号进行无线同步成为可能。

(3)\textbf{反向散射信号的双边带彼此正交:}反向散射信号的双边带拥有两个不同的频率,这两个频率远离彼此,从而反向散射信号的双边带彼此正交,可以被区分开来。这个性质使得反向散射信号的上下边带可以被分别提取出来,从而解决标签频率漂移的问题。

(4)\textbf{不同频率的反向散射信号彼此正交:}不同感知目标使用不同频移频率的标签进行反向散射调制,从而不同感知目标的反射信号彼此正交,可以被区分开来。这个性质使得多个感知目标可以同时存在而互不干扰,从而实现多目标同时感知的功能。

\subsection{信号与相量的对应关系}\label{sec:vector}

在信号处理领域中,通常使用相量\footnote{正弦量的向量表示形式。}来表示数字信号,对于形如公式~\eqref{eq:single-tone}~的单频信号,可以将其表示为一个相量:
\begin{equation}
    H = A e^{j\theta}
\end{equation}

因为信号的频率$f$决定了信号的变化速率,而在信号处理过程中,频率通常是已知的,所以可以将频率$f$作为信号的固有属性,而将幅度$A$和相位$\theta$作为信号的特征。
当接收端收到两个同频率的信号时,可以将其表示为两个相量的和,遵循向量加法规则:
\begin{equation}
    R = H_1 + H_2 = A_1 e^{j\theta_1} + A_2 e^{j\theta_2}
\end{equation}

\begin{figure}
    \centering
    \includegraphics[width=0.7\linewidth]{chap02/vector2D.pdf}
    \caption{同频率信号的相量相加}
    \label{fig:vector2D}
\end{figure}

如图~\ref{fig:vector2D}~所示,张等人的工作~\cite{zhang2021unlocking,zhang2020exploring}、谢等人的工作~\cite{xie2020combating}和江等人的工作~\cite{jiang2021sense}都采用了这种相量表示信号的方式,并且从中可以看出这些工作的局限性——由于接收端收到的信号是同频率信号的和,因此难以区分出不同的信号源,从而受限于单目标感知。

\begin{figure}
    \centering
    \includegraphics[width=0.7\linewidth]{chap02/vector3D.pdf}
    \caption{不同频率信号的相量叠加}
    \label{fig:vector3D}
\end{figure}

当接收端收到两个不同频率的信号时,可以将其认为是不同频率的相量平面的叠加。
这也对应了上一小节中提到的不同频率信号的正交性原理,因为不同频率的信号代表了不同的相量平面,所以不同频率的信号可以被区分开来。
如图~\ref{fig:vector3D}~所示,本文提出的系统正是利用了这一原理,从而使得接收端可以区分出不同感知目标的信号,进而实现多目标同时感知的功能。

\section{信号传播建模}

本节将介绍如何通过无线信号的特征(如幅度、相位等)变化来反映环境中目标的运动状态变化。
这是本工作的理论基础,当装有频移反向散射标签的目标运动状态发生变化时,经由其反射的信号的特征也会发生变化,因此可以通过分析信号特征的变化来推断目标的运动状态变化。

\begin{figure}
    \centering
    \includegraphics[width=0.9\linewidth]{chap02/propagation1.pdf}
    \caption{简单信号传播场景}
    \label{fig:propagation1}
\end{figure}

如图~\ref{fig:propagation1}~所示,不失一般性,先考虑最简单的信号传播场景,即只有一个发送端、一个接收端和一个频移反向散射标签的场景。
设光速为$c$,发送端到接收端的距离为$d$,对应信号传播时间为$\tau = \frac{d}{c}$;发送端到标签的距离为$d_1$,对应信号传播时间为$\tau_1 = \frac{d_1}{c}$;标签到接收端的距离为$d_2$,对应信号传播时间为$\tau_2 = \frac{d_2}{c}$。
设发送端与接收端的采样频率为$fs$,载波频率为$f_c$,发送端发送的基带啁啾信号为$s_0(t)=s(t)$。

\subsection{激励信号传播模型}

激励信号是指发送端发送、接收端直接接收的信号。
发送端实际发送的信号为基带信号的上变频形式:
\begin{equation}
    s_1(t) = s_0(t) \cdot e^{j 2\pi f_c t} \cdot e^{j \varphi_{\text{TX}}}
\end{equation}

其中,$\varphi_{\text{TX}}$是发送端载波的初相位。
经过时间$\tau$后,接收端收到发送端发送的信号,并下变频,得到:
\begin{equation}
    \begin{aligned}
        s_2(t) & = \alpha(\tau) \cdot s_1\left(t - \tau\right) \cdot e^{-j 2\pi f_c t} \cdot e^{-j \varphi_{\text{RX}}}                                      \\
               & = \alpha(\tau) \cdot s_0\left(t - \tau\right) \cdot e^{-j 2\pi f_c \tau} \cdot e^{j \left(\varphi_{\text{TX}} - \varphi_{\text{RX}}\right)}
    \end{aligned}
\end{equation}

其中,$\varphi_{\text{RX}}$是接收端载波的初相位,$\alpha(\tau)$是信号的路径损耗,随着传播距离的增加而减小。
接收端在接收到信号后,将对其进行时间窗口对齐,得到:
\begin{equation}
    \begin{aligned}
        s_3(t) & = s_2(t + t_\text{start})                                                                                                                                    \\
               & = \alpha(\tau) \cdot s_0\left(t + t_\text{start} - \tau\right) \cdot e^{-j 2\pi f_c \tau} \cdot e^{j \left(\varphi_{\text{TX}} - \varphi_{\text{RX}}\right)}
    \end{aligned}
\end{equation}

其中,$t_\text{start}$是接收端对齐的窗口起点位置,理想情况下,$t_\text{start} = \tau$。
最后,接收端对信号进行解调,得到激励信号:
\begin{equation}
    \begin{aligned}
        r_E(t) & = s_3(t) \cdot s_{\text{down}}(t)                                                                                                                              \\
        %    & = \alpha(\tau) \cdot e^{j 2\pi \left(f_0 + \frac{BW}{2}\right)t} \cdot e^{-j 2\pi f_c \tau} \cdot e^{j \left(\varphi_{\text{TX}} - \varphi_{\text{RX}}\right)} \\
        %    & \cdot e^{j2\pi \left(f_0 (t_\text{start} - \tau) + \frac{1}{2}k\left(2t + t_\text{start} - \tau\right)\left(t_\text{start} - \tau\right)\right)}
               & = \alpha(\tau) \cdot e^{j 2\pi \left(f_0 + \frac{BW}{2}\right)t} \cdot e^{-j 2\pi f_c \tau} \cdot e^{j \left(\varphi_{\text{TX}} - \varphi_{\text{RX}}\right)}
    \end{aligned}
\end{equation}

\subsection{反向散射信号传播模型}

如图~\ref{fig:propagation1}~所示,接收端收到的信号除了包含激励信号外,还包含有来自标签的反射信号。
发送端的信号经过时间$\tau_1$后到达标签,可以表示为:
\begin{equation}
    s_4(t) = \alpha_1(\tau_1) \cdot s_1\left(t - \tau_1\right)
\end{equation}

如前所述,标签生成方波调制入射信号,得到双边带反射信号:
\begin{equation}
    \begin{aligned}
        s_5(t) & = s_4(t) \cdot A_{\text{tag}} \cos(2\pi f_{\text{tag}} t + \psi_{\text{tag}}) \\
        %    & = \alpha_1(\tau_1) \cdot s_0\left(t - \tau_1\right) \cdot e^{j 2\pi f_c \left(t - \tau_1\right)} \cdot e^{j\varphi_{\text{TX}}}                 \\
               & = s_5^\text{upper}(t) + s_5^\text{lower}(t)
    \end{aligned}
\end{equation}

其中:
\begin{equation}
    s_5^\text{upper}(t) = \alpha'_1(\tau_1) \cdot s_0\left(t - \tau_1\right) \cdot e^{j 2\pi f_c \left(t - \tau_1\right)} \cdot e^{j\varphi_{\text{TX}}} \cdot e^{j\left(2\pi f_{\text{tag}} t + \psi_{\text{tag}}\right)}
\end{equation}
\begin{equation}
    s_5^\text{lower}(t) = \alpha'_1(\tau_1) \cdot s_0\left(t - \tau_1\right) \cdot e^{j 2\pi f_c \left(t - \tau_1\right)} \cdot e^{j\varphi_{\text{TX}}} \cdot e^{-j\left(2\pi f_{\text{tag}} t + \psi_{\text{tag}}\right)}
\end{equation}

反射信号再经过时间$\tau_2$后到达接收端,接收端下变频得到:
\begin{equation}
    \begin{aligned}
        s_6^{\text{upper}}(t) & = \alpha_2(\tau_2) \cdot s_5^{\text{upper}}\left(t - \tau_2\right) \cdot e^{-j 2\pi f_c t} \cdot e^{-j \varphi_{\text{RX}}}                                       \\
                              & = \alpha'_1(\tau_1) \cdot \alpha_2(\tau_2) \cdot s_0\left(t - \tau_1 - \tau_2\right) \cdot e^{-j 2\pi f_c \left(\tau_1 + \tau_2\right)}                           \\
                              & \quad \cdot e^{j \left(\varphi_{\text{TX}} - \varphi_{\text{RX}}\right)}  \cdot e^{j\left(2\pi f_{\text{tag}} \left(t - \tau_2\right) + \psi_{\text{tag}}\right)}
    \end{aligned}
\end{equation}
\begin{equation}
    \begin{aligned}
        s_6^{\text{down}}(t) & = \alpha_2(\tau_2) \cdot s_5^{\text{down}}\left(t - \tau_2\right) \cdot e^{-j 2\pi f_c t} \cdot e^{-j \varphi_{\text{RX}}}                                         \\
                             & = \alpha'_1(\tau_1) \cdot \alpha_2(\tau_2) \cdot s_0\left(t - \tau_1 - \tau_2\right) \cdot e^{-j 2\pi f_c \left(\tau_1 + \tau_2\right)}                            \\
                             & \quad \cdot e^{j \left(\varphi_{\text{TX}} - \varphi_{\text{RX}}\right)}  \cdot e^{-j\left(2\pi f_{\text{tag}} \left(t - \tau_2\right) + \psi_{\text{tag}}\right)}
    \end{aligned}
\end{equation}

由于捕获效应,接收端会根据激励信号的到达时间来对齐时间窗口,并进行解调,得到:
\begin{equation}
    \begin{aligned}
        r_B^\text{upper}(t) & = s_6^{\text{upper}}\left(t + t_{\text{start}}\right) \cdot s_{\text{down}}(t)                                                                                         \\
                            & \approx \alpha'_1(\tau_1) \cdot \alpha_2(\tau_2) \cdot e^{j 2\pi \left(f_0 + \frac{BW}{2} + f_{\text{tag}}\right)t} \cdot e^{-j 2\pi f_c \left(\tau_1 + \tau_2\right)} \\
                            & \quad \cdot e^{j\left(\varphi_{\text{TX}} - \varphi_{\text{RX}}\right)} \cdot e^{j\left(2\pi f_{\text{tag}} \left(\tau - \tau_2\right) + \psi_{\text{tag}}\right)}
    \end{aligned}
\end{equation}
\begin{equation}
    \begin{aligned}
        r_B^\text{lower}(t) & = s_6^{\text{lower}}\left(t + t_{\text{start}}\right) \cdot s_{\text{down}}(t)                                                                                         \\
                            & \approx \alpha'_1(\tau_1) \cdot \alpha_2(\tau_2) \cdot e^{j 2\pi \left(f_0 + \frac{BW}{2} - f_{\text{tag}}\right)t} \cdot e^{-j 2\pi f_c \left(\tau_1 + \tau_2\right)} \\
                            & \quad \cdot e^{j\left(\varphi_{\text{TX}} - \varphi_{\text{RX}}\right)} \cdot e^{-j\left(2\pi f_{\text{tag}} \left(\tau - \tau_2\right) + \psi_{\text{tag}}\right)}
    \end{aligned}
\end{equation}

这里用约等号的原因是$\tau \neq \tau_1 + \tau_2$,但是由于基带频率很低,$\tau_1$和$\tau_2$的变化很小,因此这个时间误差对基带相位的影响可以忽略不计。

\subsection{信号特征提取}\label{sec:signal_feature_extraction}

综上所述,接收端收到的信号为:
\begin{equation}
    r(t) = r_E(t) + r_B^\text{upper}(t) + r_B^\text{lower}(t)
\end{equation}

这三个信号的频率分别为$f_0 + \frac{BW}{2}$、$f_0 + \frac{BW}{2} + f_{\text{tag}}$和$f_0 + \frac{BW}{2} - f_{\text{tag}}$,相互之间频率差距远大于$\frac{1}{T}$,因此彼此正交,可以被区分开来。
通过傅里叶变换,可以得到信号幅度与相位,或者写成相量的形式:
\begin{equation}
    \begin{aligned}
        H_E & = \FFT[r(t)]\left(f_0 + \frac{BW}{2}\right)                                                                  \\
            & = \alpha(\tau) \cdot e^{-j 2\pi f_c \tau} \cdot e^{j \left(\varphi_{\text{TX}} - \varphi_{\text{RX}}\right)}
    \end{aligned}
    \label{eq:H_E}
\end{equation}
\begin{equation}
    \begin{aligned}
        H_B^\text{upper} & = \FFT[r(t)]\left(f_0 + \frac{BW}{2} + f_{\text{tag}}\right)                                                                                                       \\
                         & = \alpha'_1(\tau_1) \cdot \alpha_2(\tau_2) \cdot e^{-j 2\pi f_c \left(\tau_1 + \tau_2\right)}                                                                      \\
                         & \quad \cdot e^{j\left(\varphi_{\text{TX}} - \varphi_{\text{RX}}\right)} \cdot e^{j\left(2\pi f_{\text{tag}} \left(\tau - \tau_2\right) + \psi_{\text{tag}}\right)}
    \end{aligned}
    \label{eq:H_B_upper}
\end{equation}
\begin{equation}
    \begin{aligned}
        H_B^\text{lower} & = \FFT[r(t)]\left(f_0 + \frac{BW}{2} - f_{\text{tag}}\right)                                                                                                        \\
                         & = \alpha'_1(\tau_1) \cdot \alpha_2(\tau_2) \cdot e^{-j 2\pi f_c \left(\tau_1 + \tau_2\right)}                                                                       \\
                         & \quad \cdot e^{j\left(\varphi_{\text{TX}} - \varphi_{\text{RX}}\right)} \cdot e^{-j\left(2\pi f_{\text{tag}} \left(\tau - \tau_2\right) + \psi_{\text{tag}}\right)}
    \end{aligned}
    \label{eq:H_B_lower}
\end{equation}

这些相量的幅度和相位信息中包含了感知目标的运动信息(例如相位中的$\tau_1 + \tau_2$),通过分析这些信号特征的变化,就可以得到目标物体的运动距离和运动频率。
在多目标感知的场景中,不同的感知目标会装备不同频移的反向散射标签,并且满足$\left|f_{\text{tag}_i} - f_{\text{tag}_j} \gg \frac{1}{T}\right|$正交性条件,于是,不同目标的反向散射信号也可以被区分开来:
\begin{equation}
    r(t) = r_E(t) + \sum_{i} \left( r_{B_i}^\text{upper}(t) + r_{B_i}^\text{lower}(t) \right)
\end{equation}
\begin{equation}
    \begin{aligned}
        H_{B_i}^\text{upper} & = \FFT[r(t)]\left(f_0 + \frac{BW}{2} + f_{\text{tag}_i}\right)                                                                                                             \\
                             & = \alpha'_1(\tau_{1_i}) \cdot \alpha_2(\tau_{2_i}) \cdot e^{-j 2\pi f_c \left(\tau_{1_i} + \tau_{2_i}\right)}                                                              \\
                             & \quad \cdot e^{j\left(\varphi_{\text{TX}} - \varphi_{\text{RX}}\right)} \cdot e^{j\left(2\pi f_{\text{tag}_i} \left(\tau - \tau_{2_i}\right) + \psi_{\text{tag}_i}\right)}
    \end{aligned}
\end{equation}
\begin{equation}
    \begin{aligned}
        H_{B_i}^\text{lower} & = \FFT[r(t)]\left(f_0 + \frac{BW}{2} - f_{\text{tag}_i}\right)                                                                                                              \\
                             & = \alpha'_1(\tau_{1_i}) \cdot \alpha_2(\tau_{2_i}) \cdot e^{-j 2\pi f_c \left(\tau_{1_i} + \tau_{2_i}\right)}                                                               \\
                             & \quad \cdot e^{j\left(\varphi_{\text{TX}} - \varphi_{\text{RX}}\right)} \cdot e^{-j\left(2\pi f_{\text{tag}_i} \left(\tau - \tau_{2_i}\right) + \psi_{\text{tag}_i}\right)}
    \end{aligned}
\end{equation}

至此,本文完成了信号传播模型的建立,通过正交的激励信号和反射信号,成功提取出多个目标的反射信号特征。
值得注意的是,虽然本文在建模过程中假设了基带信号是啁啾信号,但实际上任何基带信号的感知都可以使用本文的模型进行描述,只要该信号可以解调成单频信号。
在后续的章节中,本文将详细展开该理论模型应用于实际系统时面对的挑战,并提出相应的解决方案。

\section{面向频移反向散射标签的感知模型}

上一小节的信号传播模型已经解明了无线感知的原理——信号特征变化如何对应于目标运动信息,但是实际场景中的信号传播更加复杂、需要考虑更多。
例如,实际场景中存在多径现象,同一个信号将通过不同的路径从发送端传播到接收端。
基于此,本文将继续完善信号传播模型,建立面向频移反向散射标签的感知模型。

\begin{figure}
    \centering
    \includegraphics[width=0.8\linewidth]{chap02/propagation2.pdf}
    \caption{复杂信号传播场景}
    \label{fig:propagation2}
\end{figure}

如图~\ref{fig:propagation2}~所示,场景中的信号总是能被分成两类:
一类是静态信号$r_{S_i}(t)$,包括直接从发送端发送,被接收端接收的直射信号和被静态物体反射然后被接收端接收的反射信号;
另一类是动态信号$r_{D_i}(t)$,包括由于目标运动而发生变化的反射信号。
在章节~\ref{sec:vector}~中,已经介绍了如何将信号表示为复数平面上的相量。
因此,可以把静态信号看作复数平面上不变的静态相量,而把动态信号看作复数平面上不断伸缩与转动的动态相量。

传统的感知模型收到的信号不使用频移反向散射标签区分信号,因此静态信号和动态信号都呈现以激励信号的形式,此时接收机收到的信号可以表示为:
\begin{equation}
    r(t) = \sum_{i} r_{S_i}(t) + \sum_{i} r_{D_i}(t) = r_E(t)
\end{equation}

此时,动态信号和静态信号混杂在一起,难以区分开来。
而多目标感知在这种条件下将变得更加困难,因为不同目标的动态信号也混杂在一起。
过去的工作不得不引入更多的设备支持(如多天线阵列)~\cite{zhang2021unlocking,zhang2020exploring}、更强的先验条件(如目标所在区域大致范围)~\cite{xie2020combating,xie2023boosting}、或者更复杂的信号处理算法~\cite{jiang2021sense},才能实现多目标感知。
但无论如何,这些工作都无法彻底解决动态信号和静态信号混杂的问题,并且动态信号的特征仍受到静态信号的强烈干扰。

而本文由于引入了频移反向散射标签,因此静态信号和动态信号分别呈现以激励信号和反向散射信号的形式:
\begin{equation}
    \begin{aligned}
        r(t) & = \sum_{i} r_{S_i}(t) + \sum_{i} r_{D_i}(t) \\
             & = r_E(t) + \sum_{i} r_{B_i}(t)
    \end{aligned}
\end{equation}

于是,在正交感知模型当中,所有的动态信号都是反向散射信号,可以和静态信号区分开来,而不同目标的动态信号也因为标签频移频率不同而可以被区分开来。
基于该正交感知模型,本文提出的系统无需引入额外的设备支持或先验条件,结合复杂性适度的算法,就可以实现多目标同时感知。

诚然,动态信号也存在多径的叠加,但是同一个反向散射信号仅包含一个感知目标的信息,也就是说有:
\begin{equation}
    r_{B_i}(t) = \sum_{j} r_{D_{i_j}}(t)
\end{equation}

这个问题常常在过去的工作中被忽略掉,但是本文将指出,这个问题虽然会影响目标运动幅度的感知,但并不会影响目标运动频率的感知。
后续的实验结果也说明了这一点。

\section{本章小结}

本章首先介绍了啁啾信号和反向散射信号,给出了它们的数学表达式和基本特性。
接着,本章重点介绍了信号正交的基本原理,说明了不同频率的单频信号彼此正交,可以被区分开来,由此提出了本文系统的核心理论基础。
随后,本章建立了信号传播模型,描述了如何通过分析信号特征的变化来反映目标的运动状态变化。
最后,本章进一步完善了信号传播模型,建立了面向频移反向散射标签的感知模型,并说明本文系统如何通过频移反向散射标签区分静态信号和动态信号,从而实现多目标同时感知。
为了在实际系统中应用上述理论模型,需要解决一系列挑战,下一章节中将展开介绍这些挑战,并给出相应的解决方案。
