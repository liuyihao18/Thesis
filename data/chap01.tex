% !TeX root = ../thuthesis-example.tex

\chapter{引言}

\section{研究背景与意义}

物联网,是互联网概念的延申,旨在通过各种信息传感设备将物品与物品、物品与人连接起来,实现智能识别、定位、跟踪等应用。
随着过去十几年来信息技术的飞速发展,物联网在各行各业中的应用日益广泛,涵盖了智能家居、智慧城市、智慧工业、智慧医疗等多个领域。
在国家十四五规划第十一章第一节中,明确提出了要推动物联网全面发展,打造支持固移融合、宽窄结合的物联接入能力~\cite{十四五规划}。
在今年中共二十大第四次全体会议中,也再次强调了建设数字中国的重要性~\cite{十五五建议}。
物联网的发展不仅能推动社会的数字化转型,还将为经济增长和社会进步提供新的动力。

\begin{figure}[htbp]
  \centering
  \subcaptionbox{智慧家居\label{fig:smart-home}}
  {\includegraphics[width=0.28\linewidth]{chap01/智慧家居.png}}
  \hspace{0.1\linewidth}
  \subcaptionbox{智慧城市\label{fig:smart-city}}
  {\includegraphics[width=0.28\linewidth]{chap01/智慧城市.png}}
  \linebreak
  \subcaptionbox{智慧工业\label{fig:smart-industry}}
  {\includegraphics[width=0.28\linewidth]{chap01/智慧工业.png}}
  \hspace{0.1\linewidth}
  \subcaptionbox{智慧医疗\label{fig:smart-medical}}
  {\includegraphics[width=0.28\linewidth]{chap01/智慧医疗.png}}
  \caption{物联网在各行各业中的应用}
  \label{fig:demodulation}
\end{figure}

物联网技术的重要支撑之一是无线通信技术。
随着物联网的快速发展,无线通信技术已经逐渐融入到人们的日常生活之中。
从计算机、手机等通信设备,到冰箱、洗衣机等家用电器,都实现了无线互联,极大地促进了信息的共享与交互。
然而,随着无线通信技术的广泛应用,研究者开始思考,能否进一步拓展无线技术,使无线信号不仅承担通信任务,还具备环境感知的能力?
基于这一思考,无线感知技术应运而生。
如果无线信号能够同时用于感知,那么泛在的无线信号将成为获取信息的重要途径,使得跟踪、定位等物联网应用变得更加便捷和高效。
正是在这一愿景的驱动下,无线感知技术逐渐成为学术界与工业界的研究热点~\cite{智能物联网无线定位感知关键技术研究,基于RFID的无源物联网无线感知研究现状与发展趋势,基于Wi-Fi/5G信号的无线感知与定位方法研究}。

在众多无线技术中,射频识别技术(RFID)~\cite{yang2016making, xie2020exploring,li2019towards, yang2017tagbeat, duan2018robust, wang2017poster, liang2023rf, zhao2020optimal}、无线保真(WiFi)~\cite{gui2023csi, xianjia2021just, wang2022indoor, soltanaghaei2021tagfi, kotaru2015spotfi,jiang2024willow, zeng2019farsense, hu2023muse}、超宽带(UWB)~\cite{yang2019multi, li2022fine, zheng2021more, yang2022vuloc}、调频连续波(FMCW)雷达~\cite{ahmad2018vital, islam2020non, turppa2020vital, zhang2023pi}以及可见光通信~\cite{li2015human, venkatnarayan2018gesture, an2015visible, xie2020litag}等,已经被广泛应用于各类无线感知场景,这些场景包括目标定位~\cite{kotaru2015spotfi, soltanaghaei2021tagfi, yang2022vuloc, xie2020litag, jiang2024willow}、振动检测~\cite{yang2016making, yang2017tagbeat, xie2020exploring, li2019towards}、呼吸监测~\cite{zeng2019farsense, yang2019multi, zheng2021more, li2022fine, ahmad2018vital, hu2023muse, islam2020non, turppa2020vital, zhang2023pi, li2015human}以及手势识别~\cite{venkatnarayan2018gesture, xianjia2021just, hu2023muse, xie2020litag}等。

上述无线技术所采用的信号拥有不同的载波频率和带宽,因此各自具备不同的优势和局限性。
然而,这些技术都存在感知距离受限的问题。
这一问题的根源在于,无线感知通常依赖于物体的反射信号,而反射信号相较直接发射的信号要弱得多。
对于传统通信而言,信号由发送端直接传输至接收端;而在无线感知中,信号需经由目标反射后再被接收端接收。
由于信号在传播过程中会经历路径损耗,而感知信号的传播路径长于通信信号,因此信号能量衰减更为严重,导致感知距离显著短于通信距离。
例如,WiFi的通信范围通常可达数十米,但其有效感知范围却只有约八米~\cite{zeng2019farsense}。

为了克服感知距离受限的挑战 ,研究者们尝试引入远距离通信技术LoRa(Long Range)来扩展无线感知的有效范围。
凭借LoRa优异的远距离通信能力,基于该技术的无线感知系统实现了最长约100米的感知距离~\cite{jiang2021sense}。

尽管利用LoRa技术的无线感知工作在感知距离方面取得了显著进展,但其在多目标感知方面仍存在局限。
例如,Sen-Fence~\cite{xie2020combating}与ChirpSen~\cite{xie2023boosting}系统需要在感知前预先获知目标的大致位置,这在动态环境中难以实现;
张等人~\cite{zhang2021unlocking}采用波束成形(beam forming)方法以区分多个目标,但该方法分辨率受限,尤其在远距离场景下难以有效区分彼此接近的目标;
Palantir系统~\cite{jiang2021sense}通过开关键控(OOK)反向散射标签增强感知能力,但其一次仅能处理单个标签信号。
总体而言,现有研究尚未能很好地实现对多个目标的同时感知,限制了无线感知技术的应用范围。

针对上述问题,本文提出了一种基于频移反向散射标签的远距离多目标运动特征感知系统。
该系统同样利用LoRa技术以扩展感知范围,并结合频移反向散射标签反射信号的特性构建正交感知模型,在信号处理算法上进行了创新设计,使得系统能够区分不同目标的反射信号,从而实现多目标的同时感知与特征提取。
该研究为无线感知技术在远距离多目标感知场景下的应用提供了一种可行的解决方案。

\section{国内外研究现状}

国内外利用LoRa技术进行远距离多目标感知的研究主要使用三类方法,分别是基于多天线的目标感知方法~\cite{zhang2020exploring,zhang2021unlocking}、基于区域约束的目标感知方法~\cite{xie2020combating,xie2021pushing}和基于反向散射标签的目标感知方法~\cite{jiang2021sense}。

\subsection{基于多天线的目标感知方法}

张等人在论文~\cite{zhang2020exploring}中首次提出使用LoRa技术来进行目标感知,利用LoRa技术的远距离通信能力扩展无线感知的范围。
LoRa技术是一种低功耗广域网技术,得益于线性调频扩频(CSS)的特性,能够支持大范围的通信,其信号传播距离能够达到数公里甚至数十公里。
张等人在论文中深入探讨了如何将啁啾信号(Chirp Signal)应用在无线感知应用中,\textbf{通过建立信号的传播模型},\textbf{得到了目标运动与信号相位变化之间的关系}。
作者通过数学推导,论证了通过将两根天线上接收到的信号进行相除可以消除信号相位受到的各种偏移影响,从而提高目标的感知距离。
这种形式的信号变换也被称作莫比乌斯变换~\cite{young1984linear},虽然信号相位的变化范围发生了变化,但其变化频率仍保持不变。
因此,通过对变换后的信号相位进行频率分析,仍然可以求得目标运动的频率。

这个方法虽然成功实现了对目标的感知,但是其应用场景也被限制在只能有一个感知目标在运动。
为了克服单目标感知的限制,张等人在后续的论文~\cite{zhang2021unlocking}中提出了一种基于波束成形(beam forming)的多目标感知方法。
波束成形是一种通过加权相加多根天线接收的信号来增强特定方向入射信号并抑制其他方向信号的方法。
如果想要将波束对准位于$\varphi$夹角处的目标,只需要将每根天线接收到的信号乘以一个与$\varphi$相关的权重后再相加即可。
这个权重向量常被称作导向向量(steering vector)。
通过调整导向向量,可以使得系统同时对不同方向的目标进行感知。
然而,该方法在实际应用中仍然面临一些挑战,例如天线数量会影响系统的分辨率,天线越多,波束的主瓣宽度越窄,波束对准的目标受到的干扰越小。
张等人采用了4根天线,波束的主瓣角宽度最小也有60°,如果有其他运动物体位于该范围内,目标物体的感知仍会受到干扰。
更为严重的是,这类工作致力于解决远距离感知问题,然而随着感知距离的增加,相同角宽度对应的线距离也会增加,使得目标物体更容易受到周边物体的干扰。

\subsection{基于区域约束的目标感知方法}

谢等人的工作Sen-fence~\cite{xie2020combating}在张等人的工作~\cite{zhang2020exploring}基础上,提出了一种将感知区域限制于单一感知目标并最大化目标引起的信号相位变化的方法。
值得注意的是,不同于波束成形方法,谢等人提出的方法仅需2根天线用于消除由收发端不同步引起的偏差,而无需更多的天线来进行区域限制。

谢等人将接收端收到的信号称为复合信号,并对其包含的成分进行了分类,分为与感知目标无关的静态信号和与感知目标相关的动态信号,以及可能存在的叠加在这些信号之上的偏移误差。
各种目标感知方法的实质都是在消除信号的偏移误差后,从复合信号中提取出动态信号中包含的目标运动信息。
例如张等人的工作使用多天线消除了偏移误差后,通过莫比乌斯变换的性质保证了复合信号中提取的相位变化包含动态信号中的目标运动信息。

谢等人指出,为了提高目标频率感知的精度,需要最大化复合信号的相位变化。
虽然由于目标运动导致的动态信号的相位变化是一致的,但是静态信号的相位和幅度会影响复合信号的相位变化。
当静态信号的相位和动态信号的相位相差达到180度,且和复合信号的相位相差达到90度时,复合信号的相位变化达到最大。
因此,谢等人通过引入偏置信号来改变静态信号的相位和幅度,从而最大化复合信号的相位变化。
为了求解最佳的偏置信号,谢等人设计了一种网格搜索的优化方法,并且发现如果在网格搜索时将偏置信号的相位限制在特定的区间,就可以将感知范围限制在确定的角度范围内,从而实现对角度范围内单一目标的感知。

这个方法可以在一定程度上解决多目标同时感知的问题,但仍然存在一些局限性。
首先,该方法需要提前知道感知目标的大概范围,并且该范围内只能有一个目标在运动;其次,感知目标的运动幅度不能太大,如果感知目标的运动幅度接近半个波长,则会使偏置信号的求解出现问题,令该方法失效。

除了该方法之外,谢等人还提出了另一种基于区域约束的多目标感知方法~\cite{xie2021pushing}。
与先前的方法不同,该方法创新性地利用信号的幅度变化来进行目标感知,而不是使用信号的相位变化来进行目标感知。
通常情况下,信号能量的衰减与传播距离的平方成反比,因此\textbf{目标物体的运动会引起信号幅度的变化}。
然而由于目标物体的运动幅度远小于传播距离,先前的工作都认为其对信号幅度的影响可以忽略不计。
谢等人指出,如果能够通过一些方法放大目标反射信号的能量强度,就有可能通过目标反射信号的幅度变化来对目标进行感知。
为了实现这一点,谢等人参考LoRa技术中的解调方法,设计了一种基于信号卷积的方法来汇聚一段时间内的目标反射信号的能量。
对于$N$个采样点的信号,该方法可以将目标反射信号的能量(实际上应该是信噪比)提高$N$倍,从而使目标感知距离理论上提高$\sqrt[4]{N}$倍。

在此基础上,谢等人进一步提出可以通过控制汇聚信号的窗口长度(也就是采样点的数量)来限制感知的覆盖范围,因为感知范围与采样点数量的四次方根成正比。
此时有两种情况:(1)感知目标相对于干扰目标距离收发端更近;(2)感知目标相对于干扰目标距离收发端更远。
对于第一种情况,仅需要减少窗口长度便可以控制感知范围仅覆盖感知目标;对于第二种情况,需要分别计算覆盖两个目标时的信号和只覆盖干扰目标时的信号,再对两者求差从而提取出只覆盖感知目标时的信号。

谢等人提出的两种方法成功实现了对感知区域的限制,从而实现了多目标感知功能。
然而这些方法都需要提前知道感知目标的大致位置,并且当感知目标相互靠近时无法区分出其中任何一个目标。
这些方法本质上无法很好解决多目标感知问题的原因是仅仅使用了信号的时域信息,而在多目标感知的情况下,多个目标的反射信号在时域中将完全混合起来,从而难以分离出其中任何一个目标的反射信号。
限制区域的方法通过引入一定的先验知识,增强了给定区域内的目标反射信号,并抑制了区域外的干扰反射信号,从而在一定程度上实现了多目标感知功能。
为了实现更有效的多目标感知功能,需要考虑如何获取除了时域信息外的更多信息,接下来的基于反向散射的目标感知方法将为本课题提供一些思路。

\subsection{基于反向散射的目标感知方法}

江等人的工作~\cite{jiang2021sense}受到郭等人的反向散射通信工作~\cite{guo2020aloba}的影响,试图将反向散射的远距离通信能力扩展至目标感知领域,最终实现了100米的目标感知距离。
反向散射是一种无源技术,通过反射空气中的电磁波来进行通信或感知。
RFID就是一种经典的反向散射技术,标签通过阅读器发射的电磁波获得能量,将要发送的信息编码至入射电磁波上,然后将电磁波反射回阅读器。
江等人工作的基本思想上与此相同,通过在目标物体上配置反向散射标签,增强目标物体对发送端发射电磁波的反射,然后在接收端分析反射的电磁波信号从而实现对目标物体进行感知。

与先前两类方法不同,此类方法的最大特点是可以从接收端收到的信号中\textbf{分离出感知目标的信号}。
江等人的工作使用了开关键控(OOK)标签,其会在“开”状态和“关”状态之间切换。
当标签处于“开”状态时,目标物体会反射发送端发射的信号,于是接收端收到的信号中同时包含其他路径的信号和目标物体的反射信号。
当标签处于“关”状态时,目标物体会吸收发送端发射的信号,于是接收端收到的信号中仅包含其他路径的信号。
通过将标签处于“开”状态时接收端收到的信号减去标签处于“关”状态时接收端收到的信号,就得到了目标物体的反射信号。
这类方法无需从复合信号的变化中推断目标物体的反射信号(也就是动态信号)的变化,使得目标物体的运动信息更加清晰,甚至使感知目标物体的运动距离成为可能。

江等人指出,由于感知应用对信号质量的要求远比通信应用高,所以需要特别仔细地考虑信号受到的各种偏差、误差并加以解决。
为此,江等人设计了以下信号处理模块:

(1)LoRa预处理模块:这个模块对接收端收到的信号进行必要的预处理操作,如得到啁啾信号的基本参数等。

(2)信号整形模块:这个模块首先使用一个低通滤波器来消除由硬件导致的信号幅度抖动,然后使用曲线拟合的方法去除LoRa基带信号以及各种相位偏移、漂移等。

(3)聚类模块:这个模块通过将采样点从IQ坐标系转化到幅度、相位对数坐标系,从而把复杂的双圆弧聚类问题转化为简单的双圆簇聚类问题,从而去除乘性噪声的影响。

(4)感知模块:这个模块通过动态信号的相位变化求解目标物体的运动信息,并且使用特定的滤波器来增强目标物体特定运动形式的运动信息。

江等人设计的信号处理模块很好地消除了目标反射信号中受到的各种误差,最终使目标感知达到了100米。
然而,该方法仍然存在一些问题,例如通过曲线拟合消除载波频率偏移的方法不够精确,以及聚类模块的聚类方法容易受到噪声的影响。
该方法在曲线拟合时假设了载波频率偏移等在一个啁啾信号内保持不变,但实际上这个假设并不严格成立,这使得信号整形模块最后输出的“稳定”相位还带有残余的相位变化,而这也是聚类模块提到的导致乘性噪声出现的原因之一。
在这种情况下,由于载波频率偏移的抖动是随机的,因此聚类模块得到的圆心可能代表一个啁啾符号持续时间内目标所在的任意位置。
并且由于复合信号(标签“开”状态)和静态信号(标签“关”状态)并非同时接收,两者受到的载波频率偏移很可能不一致,使得两者相减得到的动态信号仍会受到载波频率偏移的影响。

另一个问题是目标物体运动幅度的感知。
虽然江等人在论文中呈现了结果,但都有较大的误差。
这个误差来源之一是信号的二次反射,虽然多数工作中都假设目标物体的反射信号不会经过再次反射,但实际上接收端收到的目标反射信号仍然是多条反射路径叠加的结果。
这在仅对目标物体进行频率感知的情况下不成问题,但在江等人试图对目标物体进行运动幅度感知时会有较大的影响。
另外,在收发端分离的情况下,如果不知道目标物体的具体位置和运动方向,得到的目标运动幅度感知结果是否有确定的物理意义仍有待商榷。
目标物体运动幅度的精确感知仍是无线感知中领域中难以解决的问题之一。

总体而言,这篇文章对本文有较大的启发意义,既然可以从使用开关键控(OOK)反向散射标签完成通信任务~\cite{guo2020aloba}扩展到使用开关键控反向散射标签完成感知任务~\cite{jiang2021sense},也就可以从使用频移键控(FSK)反向散射标签完成通信任务~\cite{jiang2021long}扩展到使用频移键控反向散射标签完成感知任务。
如果可以通过频域信息区分出所有标签的反射信号,那么多目标感知任务将变得更加容易完成。

\section{研究内容和主要贡献}

本文旨在实现一个基于频移反向散射标签的远距离多目标运动特征感知系统。
为此,本文建立了一个面向频移反向散射标签的正交感知模型,利用反射信号的频域信息区分出不同标签的反射信号,从而实现多目标感知;
同时,该正交感知模型适应不同的基带信号类型,因此可以利用LoRa技术来扩展感知范围。
在正交感知模型的基础上,本文进一步提出了一系列信号处理方法和运动感知算法,以解决反向散射信号能量微弱、相位误差严重等问题,并提高目标运动感知的精度。
最后,本文实现了一个感知系统原型,并设计实验对其进行了充分的测试,验证了其有效性。
实验结果表明,该系统可以同时感知35个标签的运动特征,在收发端400米的感知距离下,平均频率感知误差为0.5\%。
这个感知距离是现有技术的4倍,表明本文提出的方法在远距离多目标感知方面具有优良性能。

针对上述研究背景、现状和面对的挑战,本文的主要贡献总结如下:

\subsection{正交感知模型}

先前张等人的研究工作~\cite{zhang2020exploring,zhang2021unlocking}和谢等人的的研究工作~\cite{xie2021pushing,xie2020combating}无法很好地实现多目标感知,主要原因在于仅使用了信号的时域信息,导致目标的反射信号和发送端发送的激励信号在时域中完全混合,难以分离。
虽然江等人~\cite{jiang2021sense}通过使用开关键控反向散射标签成功提取出了单个目标的反射信号,但没能做到进一步分离出多个目标各自的反射信号。
本文在江等人的工作~\cite{jiang2021long}的启发下,提出使用频移键控反向散射标签来实现这个目标。
在正交感知模型中,每个感知目标会被分配不同频率偏移的反向散射标签,这些反向散射标签将发送端发送的激励信号反射到不同的频率上,从而使得接收端可以通过信号处理方法将不同目标的信号区分开来。

\subsection{信号处理方法}

在实际场景中使用频移反向散射标签进行感知时,接收端收到的标签反射信号通常十分微弱,并且接收端收到的信号会受到多种误差的影响,例如载波频率偏移、标签频率漂移等,这些问题都会严重影响目标运动特征的感知精度。
为了解决这些问题,本文设计了一系列信号处理方法,包括信号能量集中方法、基于窗函数的旁瓣抑制方法、收发端无线同步方案以及标签频率漂移消除方法。
在信号处理的最后,本文还设计了一种综合利用信号相位和幅度进行目标运动频率感知的算法,以提高目标的感知精度。

\subsection{原型系统设计与实验验证}

为了将上述方法付诸实践,本文设计并实现了一个基于频移反向散射标签的远距离多目标运动特征感知系统原型。
这个原型系统使用商用LoRa节点或软件定义无线电设备作为发送端、软件定义无线电设备作为接收端,并制作了多个不同频率偏移的反向散射标签作为感知目标。
同时,本文还开发了带用户界面的感知程序,能够实时显示多个目标的运动特征。
通过在室内和室外环境中的大量实验,本文对该系统的性能进行了充分验证,包括感知距离、频率感知精度和多目标感知能力等方面。

\section{论文组织结构}

围绕上述研究内容,本文的组织结构安排如下:

\begin{figure}[htbp]
  \centering
  \includegraphics[width=1.0\linewidth]{chap01/论文组织结构.pdf}
  \caption{本文组织结构}
  \label{fig:sen-fence}
\end{figure}

第一章为引言,介绍了无线感知技术的应用背景与研究意义,调研了国内外相关研究现状,明确了本文的研究内容和主要贡献,并概述了论文的研究思路和行文结构。

第二章建立了正交感知模型,介绍了相关的基础知识,阐述了模型的原理和推导方法,为后续的信号处理和运动感知算法奠定了理论基础。

第三章介绍了信号处理方法,详细分析了实际应用中可能遇到的各种问题,并设计了一系列有效的处理技术,以提高目标运动特征感知的精度和可靠性。

第四章设计并实现了基于频移反向散射标签的远距离多目标运动特征感知系统原型,介绍了系统的硬件组成和软件实现,并展示了用户界面设计。

第五章对所设计的感知系统进行了全面的实验验证,评估了系统在不同环境下的性能表现,并分析了实验结果,验证了本文提出方法的有效性。

第六章为总结与展望,总结了本文的主要研究成果,指出了当前研究的不足之处,并提出了未来可能的研究方向。

