% !TeX root = ../thuthesis-example.tex

\chapter{原型系统设计}

\section{系统架构设计}

本文提出的基于频移反向散射标签的远距离多目标运动特征感知系统原型主要包含发送端、接收端以及频移标签三大硬件部分,以及算法模块和用户界面两大软件部分,其中算法模块与信号处理章节相互对应,接下来将依次展开介绍。

\section{发送端设计}

信号发送模块硬件部分使用商用Semtech SX1276 LoRa无线模块~\cite{semtech2023sx1276}或者USRP N210~\cite{etuss2023usrp}与HackRF~\cite{great2023hackrf}软件无线电设备,软件部分使用STM32 Cube IDE或者GNU Radio搭建。
Semtech SX1276 LoRa无线模块通过STM32单片机控制,上位机烧写STM32控制信号的发送;而USRP N210通过网络电缆、HackRF One通过USB与上位机连接,上位机使用GNU Radio搭建发送流图控制信号的发送。
通常情况下,USRP N210接收的信号质量更高,能得到更好的实验效果。

\section{频移标签设计}

频移反向散射标签的设计与江等人的工作~\cite{jiang2021long}类似,信号反射模块硬件部分使用射频开关ADG902~\cite{analog2023adg},通过STM32单片机控制射频开关的闭合与断开,从而实现频移反向散射标签的频率调制。
可通过电脑烧写STM32单片机程序设置不同反向散射标签的频移频率。

\section{接收端设计}

信号接收模块硬件部分使用USRP N210或HackRF One软件无线电设备,软件部分使用GNU Radio搭建。
USRP N210通过网络电缆、HackRF One通过USB与上位机进行连接,上位机使用GNU Radio搭建接收流图控制信号的接收。
接收流图如图~\ref{fig:gnu-flowgraph}~所示,
信号经由SDR接收后会分发向三个方向:
(1)发送到GNU Radio频谱分析器实时显示接收信号的性质;
(2)保存到文件以便后续处理;
(3)通过TCP传输到其他程序以进行在线信号处理与结果显示。

\begin{figure}
    \centering
    \includegraphics[width=1.0\linewidth]{chap04/gnu_flowgraph.pdf}
    \caption{GNU Radio 信号接收流图}
    \label{fig:gnu-flowgraph}
\end{figure}

\section{算法模块设计}

系统原型的算法模块设计与信号处理章节相对应,如图~\ref{fig:design}~所示,主要包含以下几个部分:

\textbf{(1)信号提取。}
该模块的目的在于提取出高信噪比(SNR)的信号,以实现精确感知。
本文在标准LoRa解调方案的基础上实现了一种共轭解调方案和一种基于概率模型的相干解调方案,以集中接收信号的全部能量。
此外,还实现了一种改进汉明窗的信号加窗方法,以缓解远近效应影响。

\textbf{(2)偏移消除。}
该模块的目的在于消除由收发端不同步以及标签频率漂移引起的相位偏移。
具体而言,本课题利用激励信号与反向散射信号之间的内在联系,通过信号相除消去了载波频率偏移(CFO)、采样时间偏移(STO)和采样频率偏移(SFO)。
随后,再结合反向散射信号双边带的信息,利用标签调制的基本特性修正了由标签频率漂移引起的误差。
除此之外,还实现了基于相邻时间窗口连续性的消除歧义方法和傅里叶变换固定相位偏移补偿方法。

\textbf{(3)目标感知。}
该模块在消除信号相位中的所有偏移和误差后,将相位转化为目标运动距离的变化,从而计算出目标的运动频率。
本文进一步采用联合估计算法,利用目标反射信号的幅度变化来增强对目标频率感知的精度。

\begin{figure}
    \centering
    \includegraphics[width=1.0\linewidth]{chap04/design.pdf}
    \caption{系统工作流程}
    \label{fig:design}
\end{figure}

\section{用户界面}

为方便用户使用和实验结果展示,本文设计了两个图形用户界面(GUI)程序,分别用于在线数据和离线数据处理。

\subsection{在线数据分析程序}

如图~\ref{fig:online-gui}~所示,在线数据分析程序的主要功能是处理分析通过TCP接收的GNU Radio发送的信号采样点数据,并实时显示目标感知信号的星座图(也就是复数平面的信号相量)和相位变化转换的运动距离变化图,以及最后分析得到的目标运动信息,如频率和振幅等。
限于界面大小,在线GUI模块最多支持两个标签的同时显示。

\begin{figure}
    \centering
    \includegraphics[width=0.75\linewidth]{chap04/online_gui.png}
    \caption{在线数据分析程序}
    \label{fig:online-gui}
\end{figure}

\subsection{离线结果分析程序}

如图~\ref{fig:offline-gui}~所示,离线结果分析程序的主要功能是对保存下来的信号采样点数据文件进行离线处理。
其中,程序左上部分是信号参数配置和标签配置,用户可以导入配置文件或者直接在界面中进行参数调整;
程序左下部分是信号处理结果显示,包括目标运动频率和运动幅度,以及一个估算的信号SNR;
程序右边部分是两个图窗,分别绘制信号相位对应的节点位置变化图以及利用联合频率估计算法得到的频率估计概率分布图。

在使用时,用户可以点击“载入数据”按钮载入保存下来的信号采样点数据,然后选择某个频率的标签,点击“运行”按钮进行信号处理并显示该标签的感知结果。
虽然该程序在每次运行时都重新处理一次数据,但实际上数据只需一次处理即可,因为一次傅里叶变换操作就可以提取出所有标签的反射信号。

\begin{figure}
    \centering
    \includegraphics[width=0.75\linewidth]{chap04/offline_gui.png}
    \caption{离线结果分析程序}
    \label{fig:offline-gui}
\end{figure}

\section{本章小结}

本章介绍了基于频移反向散射标签的远距离多目标运动特征感知系统原型的设计与实现。
系统发送端可以使用商用LoRa模块或软件无线电设备进行信号发送,频移标签通过射频开关的切换实现对激励信号的频率调制,接收端使用软件无线电设备进行信号接收。
系统算法模块包括信号提取、偏移消除和目标感知三个部分,分别对应信号处理章节的内容。
此外,系统还设计了在线和离线两种图形用户界面程序,以方便用户进行实时数据处理和离线结果分析。
