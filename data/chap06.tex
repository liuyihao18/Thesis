% !TeX root = ../thuthesis-example.tex

\chapter{总结与展望}

\section{研究总结}

本文提出了一种基于频移反向散射标签的运动特征感知系统,旨在解决远距离多目标运动感知中面临的弱信号、相位偏移和目标难以区分等挑战。
系统主要由发送端、反向散射标签和接收端组成,其工作流程包括三个核心模块:

(1)信号提取模块:通过共轭啁啾解调方法或基于概率模型的相干解调方法汇聚分散的信号能量,并利用改进汉明窗对信号进行预处理以抑制远近效应,最终从接收端采集的混合信号中提取出各个标签的反向散射信号。

(2)偏移消除模块:利用激励信号与反向散射信号之间的内在关联,消除收发端不同步引入的相位偏移(如载波频率偏移CFO、采样时间偏移STO、采样频率偏移SFO等);并基于标签调制的双边带基本特性,消除低成本标签频率漂移引入的相位误差。

(3)目标感知模块:充分融合反向散射信号的幅度与相位信息,构筑参数化轨迹优化问题,准确计算目标的运动频率等状态特征。

系统的核心在于正交感知模型——不同频率的正弦信号相互正交。
因此,只要频移反向散射标签的频率间隔满足正交性条件$\Delta f \cdot T \gg 1$,系统便可以同时处理它们的感知任务。
具体而言,系统可以区分出激励信号和标签的反向散射信号,从而使用非同步收发端相位校正方法;系统可以区分出标签反射信号的双边带,从而使用标签频率漂移消除方法;系统还可以区分出多个标签的反向散射信号,从而实现多目标同时感知。

本文提出了许多方法来实现上述模块的功能,这些方法包括:

(1)共轭啁啾解调:针对啁啾符号在非标准采样率下可能产生的频率卷绕和能量分散问题,通过生成共轭啁啾并与之相乘,使每个啁啾符号的能量集中在单个FFT频点上,从而最大化感知信号的信噪比。

(2)改进汉明窗:远近问题是指当两个信号的强度差距悬殊时,即使满足了正交性条件,强信号的旁瓣仍然有可能掩盖弱信号的主瓣。为此,本文采用了改进汉明窗函数,该窗函数额外加入直流偏置和余弦函数的三阶谐波,使得在主瓣宽度不至于增加太多的情况下旁瓣衰减额外增加了2-4.5dB。

(3)收发端无线校正:利用激励信号和反向散射信号均由同一个收发端发送而被同一个接收端接收的特性,将反向散射信号除以激励信号,从而消除掉由收发端不同步引入的相位偏移。

(4)标签频率漂移:利用反向散射信号的双边带实质是标签产生的同一个方波信号的复分解的特性,将双边带的信号相乘开平方,从而消除掉由标签频率漂移引入的相位误差。

(5)联合估计算法:与过往仅利用相位或幅度信息进行感知的方法不同,该算法构筑参数化轨迹优化问题,通过最小化幅度与相位的联合误差函数,提升最终频率估计精度。

本文进行了详细的实验以评估系统原型的性能。
通过模块性能测试,验证了各模块的有效性;
室外实验则展示了系统在不同收发端距离、标签距离、标签运动形式和干扰下的性能,其中最远感知距离达400m,频率平均感知准确率超过99.5\%,在多目标场景下,能支持35个标签的同时感知;
室内实验则展示了系统在多径环境下的性能,覆盖了20$\text{m}^2$的房间,频率平均感知准确率超过99.5\%;

总得来说,本文构建了“理论建模—方法研究—系统实现—实验验证”的完整链路,提供了一种低成本、高精度、远距离的多目标运动特征感知解决方案,为频移反向散射标签在智能家居、智慧医疗、智能安防等领域的应用奠定了基础,推动反向散射感知领域更进一步。

\section{未来展望}

本文提出的方法与系统仍然有许多值得改进和扩展的空间,未来工作可以从以下几个方面展开:

(1)硬件与标签:现有的反向散射标签原型仍然尺寸较大,为了能够更好地将标签应用于实际场景,需要进一步优化标签设计,例如将标签尺寸压缩到芯片大小,这样更方便贴于待感知物体的表面;同时,标签的供电也是需要思考的问题,未来需要考虑引入无线供电或能量采集技术,以实现标签的长期自主运行;另一方面,本文提出的正交感知模型不仅可用于啁啾信号,也可以用于其他类型的激励信号,如Wi-Fi信号等,这样可以利用更广泛的现有通信基础设施,从而提升系统的适用性和灵活性。

(2)算法:目前算法利用激励信号来校正感知信号受到的频率和时间偏移影响,但该方法不可避免地将干扰物体对激励信号的影响引入到感知信号当中,未来可以考虑参考江等人的方法~\cite{jiang2021long},通过发送跳频信号直接无线同步收发端,从而消除感知信号可能受到的影响;此外,目前的联合感知算法仅能增强往复运动目标的频率感知精度,对于更加复杂的运动模式,未来可以探寻如何利用多维度的信息来提升感知能力,这与感知应用场景密切相关,值得深入研究;最后,当前的算法仍然较为朴素,未来可以结合编解码器模型(U-Net)或自适应滤波算法等提升低SNR与强干扰下的信号处理能力,并利用循环神经网络(RNN)或者Transformer等来提升复杂运动模式下的运动特征估计精度。

(3)应用场景:本文目前的应用场景仅限于单一收发端下的目标运动频率感知,为了将应用场景扩展到更大范围,未来可以考虑引入多收发端协同工作以提升覆盖范围和感知精度;另外由于本文的正交感知模型能够支持多对收发端同时工作而互不干扰,因此可以同时获得标签不同方向的运动信息,结合收发端的位置关系,便能实现目标的运动轨迹感知,甚至实现对目标的定位,如杨等人的论文~\cite{yang2014tagoram};如此,应用场景便可以扩展到智能安防、人员定位等领域。

(4)多径问题:在扩展应用场景并将感知特征从频率信息扩展到空间位置时,室内多径问题将变得更加突出,未来研究人员必须严肃考虑这个问题。可能的解决方案包括引入定向天线来抑制多径分量、利用多天线阵列来区分不同路径的信号,或者融合使用多维度的信息克服多径带来的影响;最近的研究NeRF$^2$~\cite{zhao2023nerf2}提供了一种新的思路——利用深度学习方法来建模环境的多径特性,从而提升感知的鲁棒性和精度,这也是未来可以探索的方向。

相信未来工作的研究将为频移反向散射标签的运动特征感知系统带来更多创新与突破,推动其在各个领域的广泛应用。
